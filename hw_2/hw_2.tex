\documentclass{article}

\usepackage{amsmath}

\begin{document}
\title{Homework 2}
\author{Cyrus Ramavarapu}
\renewcommand{\today}{29 September 2016}
\maketitle

\section*{Problem 1:}
    \subsection*{A}
        Since $T_0$ cannot be parallelized and must finish before
        $T_1$, its time is fixed.  However $T_1$ can be parallelized
        over the $p$ processors give.  As a result it will take
        $\frac{T_1}{p}$ time.  The speed up can then be calculated
        by taking the ratio of the time the program will take serially
        over the time it will take if implemented in a parallel manner.
        This is given by the following equations.
        \[
            S = \frac{T_s}{T_p}
        \]
        \[
            S = \frac{t_0 + t_1}{t_0 + \frac{t_1}{p}}
        \]     
    \subsection*{B}
        Analysis of the previous speed up equation shows that the 
        time required to run a parallel implementation depends 
        inversely upon the number of processors.  As a result, limiting
        the number of processors towards positive infinity will give the
        speed up if an unlimited number of processors is available.
        \[
            \lim_{p\to\infty} S = \frac{t_0 + t_1}{t_0 + \frac{t_1}{p}}
        \]     
        \[
            S = \frac{t_0 + t_1}{t_0}
        \]     
        \[
            S = 1 + \frac{t_1}{t_0}
        \]     
    \subsection*{C}
        If the serial portion of the program, $T_0$, is defined as 
        $a = \frac{t_0}{(t_0 + t_1)}$, this value can be substituted
        into this equation and simplified.
        \[ 
            S = 1 + \frac{t_1(t_0+t_1)}{t_0}
        \]
        \[ 
            S = 1 + \frac{t_1}{a}
        \]

\section*{Problem 2:}
A deterministic algorithm for an $N\times N\times N$ mesh with wrap 
around links can be modelled on the X-Y algorithm looking at each
index individually while also considering the distance to the 
target node to determine if it should pass forward or backward.\\\\
If $x_c < x_d\ and\ abs(x_c - x_d) > x_{mid}$: send towards lower $x$\\ 
send towards higher $x$\\
If $x_c > x_d\ and\ abs(x_c - x_d) > x_{mid}$: send towards lower $x$\\ 
send towards higher $x$\\\\
If $y_c < y_d\ and\ abs(y_c - y_d) > y_{mid}$: send towards lower $y$\\
send towards higher $y$\\
If $y_c > y_d\ and\ abs(y_c - y_d) > y_{mid}$: send towards lower $y$\\
send towards higher $y$\\\\
If $z_c < z_d\ and\ abs(z_c - z_d) > z_{mid}$: send towards lower $z$\\
send towards higher $z$.\\
If $z_c > z_d\ and\ abs(z_c - z_d) > z_{mid}$: send towards lower $z$\\ 
send towards higher $z$
\section*{Problem 3:}
\subsection*{A:}
Assuming the processor and all instructions are received perfectly, the
processor can fetch $2\times10^9$ cache blocks a second.  Each cache
block is $4$ bytes.  In order to achieve maximum performance, the
memory bandwidth will have to be $4\times2\times10^9$ or $8$ Gb/s.
\subsection*{B:}
If there is no prefecthing, the cache will empty every $4$ processor
reads, or equivalently, additions.  This will lead to a delay of $80$
clock cycles while the cache line gets refilled or approximately
$20$ wasted clock cycles per addition.  Since the maximum possible
performance was $2$ GFLOPS, this gets taken down by a factor of $20$
and becomes $\frac{1}{10}^{th}$ of what it was. 
\section*{Problem 4:}
\subsection*{A:}
To map a $N\times N\ 2-D$ torus into a $N\times N\ 2-D$ mesh, a ring
can be made from each row and each column in the mesh.  The minimal
congestion will be 2 due to the extra wrap around link.  The minimal
dialation will be N-2 because of the link that wraps around reduces
the distance.
\subsection*{B:}
To map a $9$ pointed start into a $N\times N\ 2-D$ mesh, the lines
connecting the central node to the corner nodes be removed and 
replaced with two edges connecting the corner node to its 
neighbors.  The load will be $1$; however, the congestion will be
$N$ and the dialation will be $2$ because
connected to its neighbor, but corners are not connected to 
the center.
\section*{Problem 5:}
The transmission time is give by the following equation.
\[
T = I + n/b
\]
\[
T = 5 \times 10^{-7} \sec + 32 b/ 1 \times10^8 \frac{MB}{\sec}
\]
\[
T = 820\ nsec
\]

\end{document}
